Before anything may be done, a scattering cross section database must exist or be created.  Two files are included with the complete code download:
\begin{verbatim}
 ./multi_angle_mie_sizing/mie_m_code/database_files/
          scattering_coefficients_database2.mat
\end{verbatim}
and
\begin{verbatim}
 ./multi_angle_mie_sizing/mie_m_code/database_files/
          scattering_coefficients_database_all_angs.mat
\end{verbatim}
Both files are for water, with index of refraction $1.33 + 0i$.  The first is the data set that was used for the current work.  This set only contains scattering coefficients at angles important to the sizing method, but has a high angular resolution and wide size parameter range.  The second file was used to make a number of plots in the current work and contains data from angles 0-180\textdegree, but the data is only at limited resolution, and is not suitable for most sizing calculations.

For other substances, a new scattering coefficient database must be generated.  This takes a long time; the following example will include commands suitable for sending the database creation into the background where it may run uninterrupted for days.

Change to the directory containing \texttt{create\_Sx\_database.m}, and open the routine in the editor.

\begin{verbatim}
>> cd ./multi_angle_mie_sizing/mie_m_code
>> edit create_Sx_database.m 
\end{verbatim}

For this example we will create a small (but useless) database.  Set the database name (it will be created in the current directory), edit the maximum diameter, \texttt{max\_d}, to be 50$\mu$m, and the size increment \texttt{d\_inc} to be 5$\mu$m.  Edit $theta=(pi/180)*[0:5:180]$, and set the index of refraction to $m=1.448+0i$.  Now run:

\begin{verbatim}
>> create_Sx_database.m 
\end{verbatim}

In a few seconds, there should be a new database in your current directory.  If you were making a real database, it might take days, and you would want to run the process out-of-the-way in the background, like this:

\begin{verbatim}
>> system_string=strcat(...
            ['nice -n 8 /opt/matlab/bin/matlab -r "cd ''',pwd,...
            ''';create_Sx_database ;exit" &']);
>> system(system_string)
\end{verbatim}

A status file is periodically updated while this is running - at any time just:

\begin{verbatim}
>> load status_scat_calc
>> [jj calc_time(jj)]

ans =

   37.0000    4.3135

\end{verbatim}

The returned information is the current iteration number, \texttt{jj}, and the amount of total time taken up to this point, \texttt{calc\_time}.

If the database was created correctly, you should be able to run this and see the same output:

\begin{verbatim}
>> load scattering_coeff_database.mat
>> A

A = 

1x37 struct array with fields:
    S1
    S2

>> B

B = 

        m: 1.4480
        x: [1x10 double]
    theta: [1x37 double]          

\end{verbatim}

