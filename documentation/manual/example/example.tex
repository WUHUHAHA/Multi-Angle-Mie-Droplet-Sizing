\chapter{Example - Optical Method for Planar Measurement of Spray Characteristics}
\label{step_by_step_example}

The following sections present a step-by-step, Matlab-command-specific example of the entire multi-angle Mie scattering sizing method.  It will be assumed that Matlab is installed and working.  In addition, the example is executed using the linux version of Matlab; a few unix-specific commands are necessary within the sizing routines however it should be straightforward for a Windows user to make appropriate changes in these few cases.  Interested Windows users should contact the author; addition of a Windows-specific tree or improved coding allowing unrestricted use on both platforms would be welcomed.

\section{Get the Code}
\subsection{Downloading the Dissertation Code}
The Matlab routines necessary to implement the ``Multi-Angle-Mie-Sizing'' method described in the original dissertation, ``Development of a Novel Planar Mie Scattering Method for Measurement of Spray Characteristics,'' by Stephen LePera are hosted at:

https://github.com/leperas/Multi-Angle-Mie-Droplet-Sizing

The version directly referenced within the dissertation is denoted as Version 1, or v1 and is found in the download section.

WARNING!  Absolutely NO development has occurred on this version after the dissertation was finished.  It is highly recommended that if you wish to actually implement the technique, you should use the latest version from the repository!  The old dissertation code is kept available ONLY to aid in understanding and to maintain the completeness of the dissertation.

\subsection{Downloading the GitHub repository}

The actual GitHub repository is where the latest updates, capabilities, and new versions are found.  Significant bug fixes, additions and improvements have already been made since the release of the dissertation, so it is highly recommended to use the latest updated GitHub version found at:

https://github.com/leperas/Multi-Angle-Mie-Droplet-Sizing

The README within that repository should contain the most updated information about the project.

The Pro Git book, written by Scott Chacon, is highly recommended, particularly if you are unfamiliar with using the \textbf{git} version control system.

\subsection{Installing the Code}
Installation instructions are part of the code and are found in the file README.install.  The most up-to-date install and configuration information will be maintained within that file.

\subsection{Contributing to the Code}
Currently documentation consists of the original dissertation and this User Manual, and comments within the code. The goal of this User Manual is to contain the basic details of the method and up-to-date use and examples of the method as it evolves beyond the original dissertation work.

Users interested in contributing new or improved code or documentation should start by installing \textbf{git} and then getting the latest software from https://github.com/leperas/Multi-Angle-Mie-Droplet-Sizing.  Although certainly not required, users are encouraged to contact the author in order to avoid duplication, and to coordinate efforts.  

\section{Creation of the Scattering Cross Section Database}

Before anything may be done, a scattering cross section database must exist or be created.  Two files are included with the complete code download:
\begin{verbatim}
 ./multi_angle_mie_sizing/mie_m_code/database_files/
          scattering_coefficients_database2.mat
\end{verbatim}
and
\begin{verbatim}
 ./multi_angle_mie_sizing/mie_m_code/database_files/
          scattering_coefficients_database_all_angs.mat
\end{verbatim}
Both files are for water, with index of refraction $1.33 + 0i$.  The first is the data set that was used for the current work.  This set only contains scattering coefficients at angles important to the sizing method, but has a high angular resolution and wide size parameter range.  The second file was used to make a number of plots in the current work and contains data from angles 0-180\textdegree, but the data is only at limited resolution, and is not suitable for most sizing calculations.

For other substances, a new scattering coefficient database must be generated.  This takes a long time; the following example will include commands suitable for sending the database creation into the background where it may run uninterrupted for days.

Change to the directory containing \texttt{create\_Sx\_database.m}, and open the routine in the editor.

\begin{verbatim}
>> cd ./multi_angle_mie_sizing/mie_m_code
>> edit create_Sx_database.m 
\end{verbatim}

For this example we will create a small (but useless) database.  Set the database name (it will be created in the current directory), edit the maximum diameter, \texttt{max\_d}, to be 50$\mu$m, and the size increment \texttt{d\_inc} to be 5$\mu$m.  Edit $theta=(pi/180)*[0:5:180]$, and set the index of refraction to $m=1.448+0i$.  Now run:

\begin{verbatim}
>> create_Sx_database.m 
\end{verbatim}

In a few seconds, there should be a new database in your current directory.  If you were making a real database, it might take days, and you would want to run the process out-of-the-way in the background, like this:

\begin{verbatim}
>> system_string=strcat(...
            ['nice -n 8 /opt/matlab/bin/matlab -r "cd ''',pwd,...
            ''';create_Sx_database ;exit" &']);
>> system(system_string)
\end{verbatim}

A status file is periodically updated while this is running - at any time just:

\begin{verbatim}
>> load status_scat_calc
>> [jj calc_time(jj)]

ans =

   37.0000    4.3135

\end{verbatim}

The returned information is the current iteration number, \texttt{jj}, and the amount of total time taken up to this point, \texttt{calc\_time}.

If the database was created correctly, you should be able to run this and see the same output:

\begin{verbatim}
>> load scattering_coeff_database.mat
>> A

A = 

1x37 struct array with fields:
    S1
    S2

>> B

B = 

        m: 1.4480
        x: [1x10 double]
    theta: [1x37 double]          

\end{verbatim}



\section{Make an Image Data Set}

\subsection{Test the Intensity Function}
Before jumping in and making images, it is a good idea to test the intensity function, \texttt{Irr\_int.m}

Change to the directory containing \texttt{Irr\_int.m}.  Make the scattering coefficient database variables global, and load the water database that came with the downloaded code package as shown:

\begin{verbatim}
>> cd ./multi_angle_mie_sizing/mie_m_code
>> global A B
>> load ../../multi_angle_mie_sizing/database_files/
               scattering_coefficients_water_sub_angs.mat
\end{verbatim}
Define the input variables (note calculation of size factor, x):
\begin{verbatim}
>> diameter=25.02;
>> sigma=10;
>> wavelength=514.5;
>> x=2*pi*(diameter/2)*1e3/wavelength;
>> sigx=2*pi*(sigma)*1e3/wavelength;
>> PDF_type='single';
>> theta=139.03*pi()/180;
>> phi=0*pi()/180;
>> half_cone_ang=0.2*pi()/180;
>> gamma_ref=pi()/2;
>> xi=pi()/2;
>> method='full';
\end{verbatim}
Run the routine:
\begin{verbatim}
>> [intensity I Q U V matches]=Irr_int(x,sigx,PDF_type,...
         theta, phi, half_cone_ang, gamma_ref, xi, method)

intensity =

    0.3261


I =

   1.0e+03 *

         0         0    9.0955         0         0
         0    9.8185    9.8185    9.8185         0
    9.9671    9.9672    9.9673    9.9672    9.9671
         0    9.5866    9.5866    9.5866         0
         0         0    8.8292         0         0


Q =

   1.0e+03 *

         0         0   -9.0955         0         0
         0   -9.8184   -9.8185   -9.8184         0
   -9.9670   -9.9672   -9.9673   -9.9672   -9.9670
         0   -9.5866   -9.5866   -9.5866         0
         0         0   -8.8292         0         0


U =

         0         0    0.0000         0         0
         0  -25.8268    0.0000   25.8268         0
  -52.5150  -26.2580    0.0000   26.2580   52.5150
         0  -25.2935    0.0000   25.2935         0
         0         0    0.0000         0         0


V =

     0     0     0     0     0
     0     0     0     0     0
     0     0     0     0     0
     0     0     0     0     0
     0     0     0     0     0


matches =

    2.4260  152.6527

\end{verbatim}

Your output should match the above.

\subsection{Real Images}

The PNG format uses lossless compression and is an excellent choice for use with this method, as most proprietary RAW or other high-color-depth image types are easily converted with free tools such as \texttt{http://www.imagemagick.org/}.  In addition, the format enjoys full support from MATLAB.

Two possible conversion routes for real images from a Canon S90, for example, both use the \texttt{ImageMagick} program \texttt{convert} and are demonstrated below.

\begin{verbatim} 
Conversion route #1: 
(uses imagemagik which calls ufraw to do conversion)

$ convert cr2:IMG_0595.CR2 png:image.png
  
Conversion route #2:
(uses dcraw first, then imagemagik)

$ dcraw -4 IMG_0595.CR2

    then

$ convert IMG_0595.ppm image.png
\end{verbatim}

The above should only be considered a starting point - currently very little testing of real high-bit-depth images has been finished.  Be VERY wary of the effects of any conversion on the linearity of your images.

\subsection{Make Simulated Data Images}

It is useful to generate a simulated data set in order to test all the routines, and also to allow much of the computational effort in an actual experiment to be completed ahead of time.  The simulated image routine is capable of producing 8-bit and 12-bit color depth (per channel) JPEG images, and also 8-bit and 16-bit color depth (per channel) PNG images.  

For simplicity, the simulated images contain both data and registration information.  This is reflected in the .ini files by assigning the same image both as data and as registration.  In practice, it is almost impossible to get good data signal-to-noise images and at the same time capture a good image of the registration points, so two separate images taken from the exact samem location, but with different exposure/lighting, are used.

The function for making simulated images, \texttt{make\_data\_images.m}, is controlled by an initialization data file.  If no initialization file exists, the routine will create an example file.  The newly created file may be edited as desired.

For this example, change to the directory containing \texttt{make\_data\_images.m} and and run the routine as shown:
\begin{verbatim}
>> cd ./multi_angle_mie_sizing/image_creation/
>> make_data_images('new_initialization_file.ini')
\end{verbatim}

The GUI asks where to save the images; for the purposes of this example create a directory called \texttt{test\_images} and let the routine create the initialization file there.  In practice this directory and filename may be at any desired location. 

The function will now create the initialization file, and ask if the user would like to create images.  Select ``No'' if changes to the configuration are needed.  For this example, select ``Yes.''

The GUI will ask for a filename and location to save the image files.  The filename supplied will be used as the base filename for the image set; all images will be in the directory selected from the GUI.  For the example, keep the default \texttt{save\_file} name and directory and select ``Save.''

Using the default configuration as above (without any changes) will make a data set based on setup Method \#3 (MTD3), that consists of one reference image at 40\textdegree and six data images evenly spaced between 138-150\textdegree.  Droplets will cover a range between 0.5-50$\mu$m with a log-normal distribution and $\sigma=10$.  The default configuration uses a linearly vertically polarized laser as the incident light source, and a linearly oriented polarizing filter at the detector camera.

Eight figures should open.  The first seven figures are the images described above, the eighth image is a contour plot showing the simulated sizes.  The data images are saved in the chosen directory as 16-bit PNG, however the contour plot is not saved.  If you want to keep it then save it manually (but remember that all data required to recreate the contour plot is automatically saved in the chosen directory as .mat files). For each image there is a \texttt{``dg''} file and a \texttt{``In''} file, containing respectively the diameter information and intensity information.  There is also one \texttt{``info''} file which contains all the configuration parameters used by \texttt{make\_data\_images.m} to produce the simulated data set.  In addition, a template image processing initialization file is created in that directory.  The images created should look similar to those in Figure \ref{default_sim_images}.  Most printouts and screens do not have the dynamic range to show the brightest and darkest parts of the image together, however the PNG images created in this example have 16-bit color depth per channel (65536 possible intensities) and have valid pixel values for the entire size range, even in squares of the image that appear ``dark'' on the screen or in print.  In this example, the value of the darkest region is 83, compared to the 65535 highest possible value.

\begin{figure}[tbp]
\begin{center}
\includegraphics[scale=.7]{./example/make_images/droplet_sizes.eps} \\
\hspace*{1pt} \hfill (a) \hfill \hspace*{1pt} \\
\vspace*{0.1in}
\includegraphics[scale=.35]{./example/make_images/default_img_40.eps} 
%\includegraphics[scale=.3]{./example/make_images/default_img_138.eps}
%\includegraphics[scale=.3]{./example/make_images/default_img_140_4.eps}
\includegraphics[scale=.35]{./example/make_images/default_img_142_8.eps}
\includegraphics[scale=.35]{./example/make_images/default_img_145_2.eps}
\includegraphics[scale=.35]{./example/make_images/default_img_147_6.eps}
\includegraphics[scale=.35]{./example/make_images/default_img_150.eps}\\
\hspace*{40pt} 40\textdegree \hfill 142.8\textdegree \hfill 145.2\textdegree \hfill 147.6\textdegree \hfill 150\textdegree \hspace*{40pt} \\
\vspace*{0.1in}
\hspace*{1pt} \hfill (b) \hfill \hspace*{1pt} \\
\vspace*{0.1in}
\parbox{.8\linewidth}{\caption{\label{default_sim_images}(a) Plot of the simulated mean droplet sizes within the images.  From top to bottom of the image, 0.5, 6.7, 13, 19, 25, 31, 38, 44, and 50$\mu$m. (b) Simulated images at a few representative angles. }}
\end{center}
\end{figure}

The initialization file may be edited to create customized data sets.  The options for this are documented earlier in Section \ref{mie_scattering_matlab_code} and also sparsely documented within the code.  Take a few minutes to read the configuration parameters available.  

To create a data set based on an arbitrary initialization file, supply the full file name and path and re-run the routine, for instance:

\begin{verbatim}
>> make_data_images('./test/arbitary_initialization_file.ini')
\end{verbatim}


\section{Image Processing the Data Set}

A data processing initialization file, \texttt{save\_file-image\_processing.ini}, was created as part of the simulated data set in the previous section, and is located in the directory with those images.  The contents and use of this file is well documented in Section \ref{image_processing}.  

The contents may be left alone, or edited by hand.  For this example, open the file and scroll down to the \{Output\} section.  Change the parameter \texttt{output\_eps\_figures =  'Yes'} so that some images will be generated during processing.  Save changes to the file.

Change to the directory containing the function \texttt{processor\_f.m} and run the routine:

\begin{verbatim}
>> cd ./image_processing/sub_functions
>> processor_f(0)
\end{verbatim}

The processor function will open a GUI; select the data processing initialization file, \texttt{test\allowbreak-image\_\allowbreak processing.ini}.  Eleven figures will open, and a prompt to process the next angle will appear; select ``No'' and the function will close, but the figures will remain.

The following output is on the screen, and this same information has been saved into the log file, \texttt{./test\_images/processed\_output\_sim\_data/sim\_data\_processing\_log.txt}:

\begin{verbatim}
Figure origin [0  y-offset  z-offset] [ 0  0.000000  0.000000 ]  
**** Reference Image **** 
Percentage saturated pixels: 0.000000 
Actual data block limits, Left, Right, Top, Bottom: 
                        [-0.024804 0.024804 0.250196 -0.199869] 
Block height: 0.450065  width: 0.049608 
Actual super pixel square side length: 1.222222e-02 inches.
Single pixel area: 5.168952e-07 square inches.
Number of actual pixels averaged in a superpixel: 289
Total super pixel area: 1.493827e-04 square inches.
\end{verbatim}

The processor has just worked on the reference image.  Note that it's good to check that there are no saturated pixels.  If a saturation warning occurs, check to be sure the saturation is not in the data part of the image.  If other areas outside the data area are saturated, it may effect signal-to-noise but probably won't wreck the data processing.

In the processing initialization file, a sub-block of data was requested.  The exact size of the block is returned; due to the discrete number of pixels in the image this is unlikely to be exactly what was requested.  The same is true for the requested super-pixel dimensions.  

Eleven figures are opened.  Figure 1 is a histogram of the reference image; there is not a high number of ``saturation'' pixels (would be bunched at the right side of the plot) which is good.  Only the data areas have high pixel counts.

Figure 2 illustrates the image registration information.  Check that what is shown are the dimensions of the data region.

Figure 3 is original image, Figure 4 is the perspective corrected ``flat'' image, and Figure 5 is only data from the color channel containing data, in the example that is the green channel.

Figure 6 shows the image area, with the image values converted to the 0-1 range as the ``brightness matrix'' and Figure 7 is the same information, but just showing the sub-block of data requested in the initialization file.

Figure 8 and Figure 9 are the ``sister'' images of Figures 6 and 7, showing the application of the exposure time.  Because the exposure time of the reference image in this case was close to one (0.988), very little difference is observed.  

Figure 10 shows the location of the super-pixels as a ``+'' overlaying the super-pixel contour values.  Last, Figure 11 shows the same super-pixel values, but uses the actual shape of the super-pixel.  The fill-color of the super-pixels represents the value of the averaged super-pixel.

Open the data processing initialization file, \texttt{test-image\_processing.ini} again and change the \texttt{output\_eps\_figures} value back to \texttt{'No'}, then run the processor again, but with no pauses by setting the input parameter to ``2``, like this:

\begin{verbatim}
>> processor_f(2)
\end{verbatim}

The routine, now that no images are required, will very quickly re-process the reference and the other 6 data images. The on-screen output is the same in the example; in a real data set it is good to look at each image and be sure each image has processed correctly.  The intensity ratio data file has been created, \texttt{sim\_data\_ratio.mat} in the directory with all the other processing output, \texttt{./test\_images/processed\_output\_sim\_data/}.  The sizing processing initialization file is also saved there.  The data set is ready for size processing.

The data processing in this example goes very fast, but large images and large data sets may take longer.  If many data sets need to be processed, the GUI interface tool \texttt{start\allowbreak\_image\allowbreak\_processing.m} is useful.  Simply create a text file with the full path to every data image processing initialization file that needs to be processed, as many as needed, as below:

\begin{verbatim}
/full_path_to/test_images/test-image_processing.ini
/full_path_to/other_test_images/other_test-image_processing.ini
/full_path_to/more_test_images/more_test-image_processing.ini
\end{verbatim}

Edit the top of \texttt{start\allowbreak\_image\allowbreak\_processing.m} to reflect the number of processors available, and run the routine.  After using the GUI to select the above file with the list of filenames, the routine will start as many processing jobs as there are processors.  When each job finished, the routine will start another process until all the files have been processed.
 


\section{Size Processing the Data Set}

The last step is to run the size processing routine.  The sizing processing initialization file was created and stored in the same directory with the other image processing results.  Options within the sizing initialization file are covered within Section \ref{sizing_calcuations}, however the file created in this example may be used as-is.

Change to the directory with the size processing function \texttt{analyze\_sizes\_f.m}, and run the \texttt{analyze\_sizes\_f.m} function.

\begin{verbatim}
 >> cd ./multi_angle_mie_sizing/sizing_functions/sub_functions
 >> [best_mean best_std best_sigma best_dist match_fraction conf_num]=...
                      analyze_sizes_f(0)
\end{verbatim}

Use the GUI to select the sizing initialization file created by the data processing routine, \texttt{sim\_data\_sizing\_ini.ini}.  After 30 seconds or so, the output file \texttt{.../test\allowbreak\_images/\allowbreak processed\allowbreak\_output\allowbreak\_sim\_\allowbreak data/\allowbreak sizing/\allowbreak output.mat} will be created, and the following returned to the screen:

\begin{verbatim}
 best_mean =

  305.3284
  268.2088
  231.9101
  192.7971
  157.0158
  115.8034
   81.7921
  305.3064
         0


best_std =

    0.0408
    0.9991
    0.4693
    0.8588
    3.3571
    0.9926
    2.0215
    0.0088
         0


best_sigma =

   10.0000
    6.5000
   10.0000
   20.0000
    6.5000
   10.0000
    6.5000
   10.0000
    4.8000


best_dist = 

    'logn'
    'logn'
    'logn'
    'logn'
    'logn'
    'logn'
    'logn'
    'logn'
    'norm'


match_fraction =

    1.0000
    1.0000
    0.8333
    0.6667
    1.0000
    0.8333
    0.8333
    0.5000
    0.1667


conf_num =

    1.0000
    1.0000
    0.6554
    0.5837
    1.0000
    0.8565
    0.6555
    0.2249
    0.0574

\end{verbatim}


The \texttt{best\_mean} is the final size output at each location in the region requested by the sizing initialization file, in this case the middle ``column'' of the image.  The output is summarized as below (after converting the size parameter back to diameter).


\begin{tabular}{cccccccccc}
\\ \hline
Mean, $\mu$m  & 0.0 & 50.0 & 13.4 & 19.0 & 25.7 & 31.6 & 38.0 & 43.9 & 50.0 \\ 
std dev  & 0.00 & 0.00 & 0.66 & 0.33 & 1.10 & 0.28 & 0.15 & 0.33 & 0.01 \\ 
 $\sigma$  & 4.8 & 10.0 & 6.5 & 10.0 & 6.5 & 20.0 & 10.0 & 6.5 & 10.0 \\ 
Distribution & norm & logn & logn & logn & logn & logn & logn & logn & logn \\ 
 CN  & 0.06 & 0.22 & 0.66 & 0.86 & 1.00 & 0.58 & 0.66 & 1.00 & 1.00 \\ 
Error \%  & 100.00 & 601.76 & 3.04 & 0.18 & 1.10 & 1.33 & 0.05 & 0.17 & 0.01 \\ \hline 
\end{tabular}

