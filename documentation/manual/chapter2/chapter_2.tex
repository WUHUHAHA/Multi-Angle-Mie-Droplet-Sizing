\chapter{The Experimental Method}
\label{chapter_exp}
\section{Optical Method for Planar Measurement of Spray Characteristics using Mie Scattering Theory}

The theory and background required to build the foundation of a planar (2D) optical spray characterization method was presented in Section \ref{background}.  From that discussion it is clear that information about the mean droplet size,  and the width and shape of of a spray size distribution is encoded within the distribution of intensities scattered from droplets within a particular plane of interest.  With the Mie theory in hand, if the characteristics of a spray are known then this distribution of scattered intensities may be calculated.  

The following sections address the ``reverse problem'' where scattered intensities at a discrete number of locations are measured, and then this information is used to deduce the size and size distribution of unknown droplets within the planar interrogation region.

The procedure used here is as follows:
\begin{enumerate}
\item The spray of droplets must be located within a specific optical setup and the geometry of all optical and experimental components must be well known, controlled, and placed in a specifically designed layout conducive to collection of the most information from the least number of discrete measurements.  The region of interest is illuminated by a collimated, planar laser sheet.
\item Images from multiple angular locations are taken of the region of interest, either simultaneously or by moving a single detector during a continuous (stationary) experiment.  A minimum of two measurements are required - one reference angle and one data angle.  Additional data angles provide improvements in mean droplet diameter determination and add information about the droplet size distribution.
\item After the experiment is complete, the images are ``registered'' spatially into a common coordinate system, corrected for perspective and detector non-linearity, and processed to produce ``intensity ratios'' between data angles and a reference angle.  This ratio is therefore available at every point/area within the planar region of interest.
\item The measured intensity ratios are compared against known Mie theory scattering intensity ratios to deduce the size and size distribution of the droplets at all points/areas within the planar region of interest.
\end{enumerate}

The following sections detail the ``nuts and bolts'' of the current technique.  A complete step-by-step example of the above procedure is presented in Appendix \ref{step_by_step_example}.

\section{Optical Diagnostics}

%\input{./chapter2/optical_diagnostics/optical_diagnostics.tex}

\section{Data Processing}
\label{data_processing}

  \subsection{Computation of Mie Theory Scattering Functions}
  \label{mie_scattering_matlab_code}

 % \input{./chapter2/data_processing/data_image_creation/computation_mie_scatter_funcs.tex}
 % \input{./chapter2/data_processing/data_image_creation/simulated_scattering_image_generation.tex}

  \subsection{Image Processing}
  \label{image_processing}

 % \input{./chapter2/data_processing/image_processing/initialization_file/initialization_file.tex}
 % \input{./chapter2/data_processing/image_processing/processor_function/processor_function.tex}

  \subsection{Sizing Calculations}
  \label{sizing_calcuations}

 % \input{./chapter2/data_processing/sizing/sizing_ini_file.tex}
 % \input{./chapter2/data_processing/sizing/sizing_function.tex} 
